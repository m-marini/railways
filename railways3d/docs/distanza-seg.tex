\documentclass[10pt,a4paper]{article}
\usepackage[utf8]{inputenc}
\usepackage{amsmath}
\usepackage{amsfonts}
\usepackage{amssymb}
\usepackage{makeidx}
\usepackage{graphicx}
\title{Distanza punto segmento}
\begin{document}

Siano $ \overline{AB}, O $ rispettivamente il segmento che congiunge i punti $ A, B $ e $ O $ un punto qualsiasi nello spazio.

La retta $ r $ passante per $ \overline{AB} $ è definita
\[
   x^i = ( x_B^i - x_A^i ) t + x_A^i 
\]

Sia $ H $ il punto di intersezione tra la retta $ r $ e la retta passante per $ O $ e $ \perp $ a  $ r $.

Per calcolare la distanza $ d $ tra $ O $ e il segmento $ \overline{AB} $ dobbiamo
sapere dapprima se $ H \in AB $. Ciò e vero se
\[
0 \le \frac{\vec{OA} \cdot \vec{BA}} {|| \overline{BA} || ^ 2} \le 1
\]

In tal caso
\begin{equation}
	d = \sqrt{|| \overline{OB} || ^ 2 - || \overline{AH} || ^ 2}
	  = \sqrt{|| \overline{OB} || ^ 2
	  - \frac{(\vec{OA} \cdot \vec{BA}) ^ 2} {|| \overline{BA} || ^ 2}}
\end{equation}

Se invece 
\[
\frac{\vec{OA} \cdot \vec{BA}} {|| \overline{BA} || ^ 2} < 0
\]
abbiamo
\begin{equation}
	d = || \overline{OA} ||
\end{equation}

Nell'ultimo caso 
\[
\frac{\vec{OA} \cdot \vec{BA}} {|| \overline{BA} || ^ 2} > 1
\]
abbiamo
\begin{equation}
	d = || \overline{OB} ||
\end{equation}

\end{document}