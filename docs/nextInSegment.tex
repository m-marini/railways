\section{Calcolo del punto di coda di una carrozza in un binario lineare}

Siano $(x,y)$ le coordinate della testa della carrozza,
$l$ la lunghezza del binario,
$(x_0,y_0)$ il punto iniziale del segmento e
$\alpha$\ la direzione del segmento rispetto al nord.
\\

L'equazioni del segmento sono:  
\begin{equation}
  \label{eq:no1}
  \begin{array}{l}
    x=x_0+s\sin\alpha
    \\
    y=y_0-s\cos\alpha
  \end{array}
\end{equation}
per
\begin{eqnarray}
  \label{eq:no2}
  s\ge 0
  \\
  \label{eq:no3}
  s\le l
\end{eqnarray}

Sia ora un punto $P=(x_p,y_p)$, calcoliamo le coordinate dei punti,
appartenenti al segmento, distanti $d$ dal punto $P$:
\begin{displaymath}
  \begin{array}{l}
    d^2=(x-x_p)^2+(y-y_p)^2
    \\
    (x_0-x_p+s\sin\alpha)^2+(y_0-y_p-s\cos\alpha)^2-d^2=0
    \\
    s^2
    +2s[(x_0-x_p)\sin\alpha-(y_0-y_p)\cos\alpha]
    +(x_0-x_p)^2+(y_0-y_p)^2-d^2=0
  \end{array}
\end{displaymath}
posto
\begin{eqnarray}
  b=[(x_0-x_p)\sin\alpha-(y_0-y_p)\cos\alpha]
  \\
  c=(x_0-x_p)^2+(y_0-y_p)^2-d^2
\end{eqnarray}
abbiamo
\begin{displaymath}
  s^2+2bs+c=0
\end{displaymath}
da cui
\footnote{
  Verifica
  \begin{displaymath}
    \begin{array}{l}
      s^2=b^2+b^2-c\mp2b\sqrt{b^2-c}=2b^2-c\mp2b\sqrt{b^2-c}
      \\
      (2b^2-c\mp2b\sqrt{b^2-c})+2b(-b \pm\sqrt{b^2-c})+c=
      \\
      =2b^2-c\mp2b\sqrt{b^2-c}-2b^2\pm2b\sqrt{b^2-c}+c=0
    \end{array}
  \end{displaymath}
}
\begin{eqnarray}
  \Delta=b^2-c
  \\
  \label{eq:no4}
  s=-b\pm\sqrt\Delta
\end{eqnarray}

\section{Analisi}

La (\ref{eq:no4}) ammette due soluzione
\begin{eqnarray}
  \label{eq:no5}
  s=-b+\sqrt\Delta
  \\
  \label{eq:no6}
  s=-b-\sqrt\Delta
\end{eqnarray}

Analizziamo le varie casistiche:

\subsection{Per $\Delta=0$}

Abbiamo allora una sola soluzione

\begin{equation}
  s=-b
\end{equation}
che per soddisfare la (\ref{eq:no2}) \`e necessario che $b \le 0$.

Mentra per la (\ref{eq:no3}), che sostanzialmente \`e la condizione per
verificare se il punto \`e contenuto nel segmento piuttosto che nella tratta
sucessiva, abbiamo:
\begin{equation}
  b\ge-l
\end{equation}

\subsection{Per $\Delta>0$ e $b<0$}

\paragraph{Condizioni per la soluzione (\ref{eq:no5})}
Essendo $b$ negativo sicuramente dalla (\ref{eq:no5}) $s$ \`e positivo e
quindi la (\ref{eq:no2}) \`e soddisfatta.

Rimane per cui $-b+\sqrt\Delta\le l$ quindi
\begin{equation}
  \sqrt\Delta\le l+b
\end{equation}

\paragraph{Condizioni per la soluzione (\ref{eq:no6})}

Dalla (\ref{eq:no2}) abbiamo che $-b-\sqrt\Delta\ge 0$ e quindi
\begin{displaymath}
  \sqrt\Delta\le -b    
\end{displaymath}

Mentre per la (\ref{eq:no3}) abbiamo che $-b-\sqrt\Delta\le l$ da cui

\begin{equation}
  \sqrt\Delta\ge -b-l
\end{equation}

Nei casi reali l'ambiguit\`a si genera solo quando il punto di riferimento
$P$ appartiene al segmento stesso e quindi il valore maggiore delle due
soluzioni, la (\ref{eq:no5}), \`e il punto sucessivo nel segmento.

L'altro caso di ambiguit\`a \`e causato da una tratta curva con
raggio di curvatura minore della lunghezza della carrozza, caso che per le
caratteristiche delle tratte curve e delle carrozze non \`e possibile.

\subsection{Per $\Delta>0$ e $b\ge0$}

Essendo $b$ positivo la (\ref{eq:no6}) non pu\`o soddisfare la
(\ref{eq:no3}), quindi rimane valida solo la soulzione (\ref{eq:no5})
alla condizione $-b+\sqrt\Delta\ge 0$ e
quindi
\begin{displaymath}
  \sqrt\Delta\ge b
\end{displaymath}
e $-b+\sqrt\Delta\le l$
ovvero
\begin{equation}
  \sqrt\Delta\le b+l      
\end{equation}

Riassumento quindi l'unica soluzione realmente valida rimane la
(\ref{eq:no5}) alla verifica della condizione imposta dalla (\ref{eq:no3}).
