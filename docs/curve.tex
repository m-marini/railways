\section{Calcolo del punto in un binario curvo}
Sia $O=(x_0, y_0)$ il punto iniziale del binario, $r$ il raggio di curvatura
del binario, $v$ il verso della curvatura con $v=1$ se il binario curva a
destra e $v=-1$ se il binario curva a sinistra e $\alpha$ la direzione del
binario nel punto $O$.

Avremo allora che il luogo dei punti appartenenti al binario \`e:
\begin{equation}
  \label{eq:curve1}
  \left|
  \begin{array}{l}
    x=x_c+r\sin\beta
    \\
    y=y_c-r\cos\beta
  \end{array}
  \right.
\end{equation}
dove $\beta=\alpha+v(\frac{s}{r}-\frac{\pi}{2})$.

Applicando le formule di somma di angoli abbiamo che
\begin{displaymath}
  \begin{array}{l}
    \sin\beta=\sin(\alpha+v\frac{s}{r}-v\frac{\pi}{2})=
    \\
    =\sin(\alpha+v\frac{s}{r})\cos(v\frac{\pi}{2})-
    \cos(\alpha+v\frac{s}{r})\sin(v\frac{\pi}{2})=
    \\
    =-v\cos(\alpha+v\frac{s}{r})
  \end{array}
\end{displaymath}
mentre
\begin{displaymath}
  \begin{array}{l}
    \cos\beta=\cos(\alpha+v\frac{s}{r}-v\frac{\pi}{2})=
    \\
    =\cos(\alpha+v\frac{s}{r})\cos(v\frac{\pi}{2})+
    \sin(\alpha+v\frac{s}{r})\sin(v\frac{\pi}{2})=
    \\
    =v\sin(\alpha+v\frac{s}{r})
  \end{array}
\end{displaymath}
quindi la (\ref{eq:curve1}) diventa
\begin{equation}
  \label{eq:curve2}
  \left|
  \begin{array}{l}
    x=x_c-vr\cos(\alpha+v\frac{s}{r})
    \\
    y=y_c-vr\sin(\alpha+v\frac{s}{r})
  \end{array}
  \right.
\end{equation}
Posto che per $s=0$ abbiamo $x=x_0$, $y=y_0$ da cui
\begin{displatmath}
  \left|
  \begin{array}{l}
    x_0=x_c-vr\cos\alpha
    \\
    y_0=y_c-vr\sin\alpha    
  \end{array}
  \right.
\end{displatmath}
quindi
\begin{displaymath}
  \left|
  \begin{array}{l}
    x_c=x_0+vr\cos\alpha
    \\
    y_c=y_0+vr\sin\alpha    
  \end{array}
  \right.
\end{displaymath}
La (\ref{eq:curve2}) diventa allora
\begin{equation}
  \left|
  \begin{array}{l}
    x=x_0-vr[\cos(\alpha+v\frac{s}{r})-\cos\alpha]
    \\
    y=y_0-vr[\sin(\alpha+v\frac{s}{r})-\sin\alpha]
  \end{array}
  \right.
\end{equation}
